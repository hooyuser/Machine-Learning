\documentclass{article}
% Comment the following line to NOT allow the usage of umlauts
\usepackage[utf8]{inputenc}
% Uncomment the following line to allow the usage of graphics (.png, .jpg)
\usepackage{geometry}
\geometry{left=3cm,right=3cm,top=3cm,bottom=3cm}

\usepackage[usenames,dvipsnames]{color}
\usepackage[colorlinks,linkcolor=NavyBlue,anchorcolor=red,citecolor=green]{hyperref}
\usepackage{graphicx}
\usepackage{enumerate}
\usepackage{multirow}
\usepackage{float}
\usepackage{bm}
\usepackage{array,makecell}
\usepackage[table,xcdraw]{xcolor}
\usepackage{amsmath,amsfonts}
\usepackage[thmmarks,amsmath]{ntheorem}
\theorembodyfont{\upshape}
\newtheorem{definition}{Definition}[section]
\newtheorem{example}{Example}[section]
\newtheorem{proposition}{Proposition}[section]
\newtheorem{theorem}{Theorem}[section]
\newtheorem{lemma}{Lemma}[section]
\theoremstyle{nonumberplain}
\theoremheaderfont{\itshape}
\theorembodyfont{\normalfont}
\theoremsymbol{\\ \rightline{$\square$}}
\newtheorem{proof}{Proof.}

% Start the document
\begin{document}
	\begin{center}
	\textsc{\Huge Machine Learning}	
\end{center}
\vspace{1em} 

% Create a new 1st level heading
\section{Terminology}

\begin{itemize}
	\item Data generating process: $(X,Y)$ is a $(p+1)-\text{dimensional}$ random vector with joint distribution $\mathrm{P}(x,y)$.
	\begin{itemize}
		\item Input vector: $X\in D\subset\mathbb{R}^p$.
		\item Output vector: $Y\in E\subset\mathbb{R}$.
		\item Data: Given the sample $\{(X_1,Y_1),(X_2,Y_2),\cdots,(X_N,Y_N)\}$ following the distribution $\mathrm{P}(x,y)$, The training data or text data $T=\{(x_1,y_1),(x_2,y_2),\cdots,(x_N,y_N)\}$ consist of the realization values of the sample.
	\end{itemize}	
	\item Objective: Find a decision function $\hat{f}$ to minimize the expected loss
	\[
	\min_{\hat{f}\in\mathcal{F}}\mathrm{E}(L(Y,\hat{f}(X)))
	\]
	\begin{itemize}
		\item Decision function: $\hat{f}:\mathbb{R}^p\supset D\longrightarrow\mathbb{R}$ serves to produce the prediction $\hat{y}=\hat{f}(x)$ of $Y$, provided a specified value $x$ of $X$. 
		\item Loss function: $L(Y,\hat{f}(X))$ normally has the form of 
		\[
		L_2=(Y-\hat{f}(X))^2\ \text{ or }\ L_1=|Y-\hat{f}(X)|.
		\]
		\item Hypothesis space: $\mathcal{F}$ is a collection of all decision functions $\hat{f}$ to be selected. In some cases, we suppose that as a candidate $\hat{f}$ can be specified by several parameters. Thus $\mathcal{F}=\{\hat{f}_\theta:Y=\hat{f}_\theta(X),\;\theta\in\mathbb{R}^n\}$ can be described by the parametric space $\Theta=\{\theta:Y=\hat{f}_\theta(X),\;\theta\in\mathbb{R}^n\}$.
	\end{itemize}
	\item Optimization strategies: \\
	empirical risk minimization:
	\[
	\min_{\hat{f}\in\mathcal{F}}\frac{1}{N}\sum_{i=1}^{N}L(y_i,\hat{f}(x_i))
	\]
	structural risk minimization:
	\[
	\min_{\hat{f}\in\mathcal{F}}\frac{1}{N}\sum_{i=1}^{N}L(y_i,\hat{f}(x_i))
	\]
\end{itemize} 





\end{document}
