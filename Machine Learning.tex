\documentclass{report}
% Comment the following line to NOT allow the usage of umlauts
\usepackage[utf8]{inputenc}
% Uncomment the following line to allow the usage of graphics (.png, .jpg)
\usepackage{geometry}
\geometry{left=3cm,right=3cm,top=3cm,bottom=3cm}

\usepackage[usenames,dvipsnames]{color}
\usepackage[colorlinks,linkcolor=NavyBlue,anchorcolor=red,citecolor=green]{hyperref}
\usepackage{graphicx}
\usepackage{enumerate}
\usepackage{multirow}
\usepackage{float}
\usepackage{bm}
\usepackage{array,makecell}
\usepackage[table,xcdraw]{xcolor}
\usepackage{amsmath,amsfonts}
\usepackage[thmmarks,amsmath]{ntheorem}
\theorembodyfont{\upshape}
\newtheorem{definition}{Definition}[section]
\newtheorem{example}{Example}[section]
\newtheorem{proposition}{Proposition}[section]
\newtheorem{theorem}{Theorem}[section]
\newtheorem{lemma}{Lemma}[section]
\theoremstyle{nonumberplain}
\theoremheaderfont{\itshape}
\theorembodyfont{\normalfont}
\theoremsymbol{\\ \rightline{$\square$}}
\newtheorem{proof}{Proof.}

% Start the document
\begin{document}
	\begin{center}	
	\vspace{1em} 
	\textsc{\Huge Machine Learning}	
	\end{center}


% Create a new 1st level heading
\chapter{Introduction}

\section{Terminology and framework}

\begin{itemize}
	\item Data generating process: $(X,Y)$ is a $(p+1)-\text{dimensional}$ random vector with joint distribution $\mathrm{P}(x,y)$.
	\begin{itemize}
		\item Input vector: $X\in D\subset\mathbb{R}^p$.
		\item Output vector: $Y\in G\subset\mathbb{R}$.
		\item Data: Given the sample $\{(X_1,Y_1),(X_2,Y_2),\cdots,(X_N,Y_N)\}$ following the distribution $\mathrm{P}(x,y)$, The training data or text data $T=\{(x_1,y_1),(x_2,y_2),\cdots,(x_N,y_N)\}$ consist of the realization values of the sample.
	\end{itemize}	
	\item Objective: Find a optimal decision function $\hat{f}$ to minimize the expected prediction loss (EPE)
	\[
	\min_{f\in\mathcal{F}}\mathrm{E}[L(Y,f(X))]
	\]
	\begin{itemize}
		\item Decision function: $f:\mathbb{R}^p\supset D\longrightarrow\mathbb{R}$ serves to produce the prediction $f(x)$ of $Y$, provided a specified value $x$ of $X$. 
		\item Loss function: $L(Y,f(X))$ normally has the form of 
		\[
		L_2=(Y-f(X))^2\ \text{ or }\ L_1=|Y-f(X)|\ \text{ or }\ L_I=1_{Y\ne f(X)}.
		\]
		\item Hypothesis space: $\mathcal{F}$ is a collection of all potential decision functions $f$ to be selected. In some cases, we suppose that $f$ as a candidate can be specified by several parameters. Thus $\mathcal{F}=\{f_\theta:Y=f_\theta(X),\;\theta\in\mathbb{R}^n\}$ can be described by the parametric space $\Theta=\{\theta:Y=f_\theta(X),\;\theta\in\mathbb{R}^n\}$.
	\end{itemize}
	\item Optimization strategies: \\
	empirical risk minimization:
	\[
	\min_{f\in\mathcal{F}}\frac{1}{N}\sum_{i=1}^{N}L(y_i,f(x_i))
	\]
	structural risk minimization:
	\[
	\min_{f\in\mathcal{F}}\frac{1}{N}\sum_{i=1}^{N}L(y_i,f(x_i))+\lambda J(\hat{f})
	\]
\end{itemize} 
\section{Squared error loss}
\subsection{Quantitative output variables}
Let $Y\in\mathbb{R}$ be quantitative variable. And we take the most
common and convenient loss function, squared error loss
\[
L(Y,f(X))=(Y-f(X))^2.
\]That leads to the problem of minimizing the expected prediction error
\[
\min_{f\in\mathcal{F}}\mathrm{E}[(Y-f(X))^2]=\min_{f\in\mathcal{F}}\int (y-f(x))^2\;\mathrm{dP}(x,y).
\]
Note that
\[\mathrm{E}[(Y-f(X))^2]=\mathrm{E}[\mathrm{E}[(Y-f(X))^2|X]]=\int\mathrm{E}[(Y-f(x))^2|X=x]\;\mathrm{dP}(x).\]
It suffices to minimize EPE pointwise, that is, 
\begin{align*}
&\ \ \min_{f(x)\in\mathbb{R}}\mathrm{E}[(Y-f(x))^2|X=x]\\
&=\min_{c\in\mathbb{R}}\mathrm{E}[(Y-c)^2|X=x]\\
&=\min_{c\in\mathbb{R}}\mathrm{Var}[Y-c|X=x]-(\mathrm{E}[(Y-c)|X=x])^2\\
&=\min_{c\in\mathbb{R}}\mathrm{Var}[Y|X=x]-(\mathrm{E}[Y|X=x]-c)^2.\\
\end{align*}
We see the optimal solution is
\[
\hat{f}(x) = \mathrm{E}[Y|X=x],
\]
Thus the best prediction of $Y$ at any point $X = x$ is the conditional expectation, when best is measured by average squared error. \\
Next we are developing effective methods to estimate the conditional expectation $\mathrm{E}[Y|X=x]$.
\subsection{Categorical output variable}
Assume that $Y\in G$ is a categorical variable and that the set of possible classes $G=\{G_1,G_2,\cdots,G_K\}$. This time the 0–1 loss function 
\[
L(Y,\hat{f}(X))=1_{Y\ne\hat{f}(X)}=
\begin{cases}
0,&Y=\hat{f}(X),\\
1,&Y\ne\hat{f}(X),
\end{cases}
\]
is adopted for simplification. Likewise it suffices to minimize EPE pointwise.
\begin{align*}
&\ \ \min_{\hat{f}(x)\in G}\mathrm{E}[1_{Y\ne\hat{f}(x)}|X=x]\\
&=\min_{g\in G}\mathrm{E}[1-1_{Y=g}|X=x]\\
&=\min_{g\in G}1-\mathrm{P}(Y=g|X=x)
\end{align*}
And the optimal solution is
\[
\hat{f}(x)=\max_{g\in G} \mathrm{P}(Y=g|X=x)
\]
\section{Nearest-neighbor methods}
Nearest-neighbor methods use those observations in the training set $T$ closest in input space to $x$ to estimate the aforementioned conditional expectation  $\mathrm{E}[Y|X=x]$. Specifically, the $k$-nearest neighbor fit for $\hat{Y}$ is defined as follows:
\[
\hat{Y}=\hat{f}(x)=\dfrac{1}{k}\sum_{i:x_i\in N_k(x)}y_i,
\]
where $N_k(x)$ is the neighborhood of $x$ defined by the $k$ closest points $x_i$ in the training sample. Closeness implies a metric, which for the moment we assume is Euclidean distance.


\end{document}
